\documentclass[12pt]{article}
\usepackage[utf8]{inputenc}
\usepackage{amsmath,amsthm,amsfonts,amssymb,amscd}
\usepackage[table]{xcolor}
\usepackage[margin=2cm]{geometry}
\usepackage{graphicx}
\usepackage{multicol}
\usepackage{mathtools}
\usepackage[T1]{fontenc}
\usepackage{Alegreya}
\usepackage[brazil]{babel}
\newlength{\tabcont}
\setlength{\parindent}{0.0in}
\setlength{\parskip}{0.05in}

\begin{document}
	
	\textbf{Nome}: Luís Felipe de Melo Costa Silva \\
	\textbf{Número USP}: 9297961 
	
	\begin{center}
		\LARGE \bf
		Lista de Exercícios 8 - MAE0228
	\end{center}
	 
	\section*{Exercício 4}
	
	\textbf{a)}  $ \{X_t; t\geq0\} $ é uma cadeia de Markov em tempo contínuo porque pode ser modelada como um processo de nascimento e morte. Sua taxa de nascimento é $ \lambda_i = \lambda $, para $ i = 0, 1, 2, ... $ e a taxa de morte é $ \mu_i = \mu $, para $ i = 1, 2, 3, ... $. O espaço de estados é $ S=\{0, 1, 2, 3, ...\} $ e seu gerador infinitesimal G é:
	
	\begin{center}
		$
		G = \left[\begin{array}{ccccccc}
		          -\lambda  & \lambda          & 0                & 0                & 0       & 0   & ... \\
		          \mu       &  -(\lambda+\mu)  & \lambda          & 0                & 0       & 0   & ... \\
		          0         & \mu              &  -(\lambda+\mu)  & \lambda          & 0       & 0   & ... \\
		          0         & 0                & \mu              &  -(\lambda+\mu)  & \lambda & 0   & ... \\ 
		          \vdots       & \vdots              & \vdots              &  \vdots             & \vdots     & \vdots & \vdots \\
		    \end{array}\right]
		$
	\end{center}
	
	\textbf{b)} Para calcularmos a distribuição estacionária, vamos fazer $ \pi G = 0 $:\\
	
	$ \begin{cases}
		-\lambda \pi_0 + \mu \pi_1 = 0 \to \pi_1 = \frac{\lambda}{\mu} \pi_0\\
		\lambda \pi_0 -(\lambda+\mu) \pi_1 + \mu \pi_2 = 0 \to \pi_2 = \left(\frac{\lambda}{\mu}\right)^2 \pi_0\\
		\lambda \pi_1 -(\lambda+\mu) \pi_2 + \mu \pi_3 = 0 \to \pi_3 = \left(\frac{\lambda}{\mu}\right)^3 \pi_0\\
		\vdots \\
		\lambda \pi_{k-2} -(\lambda+\mu) \pi_{k-1} + \mu \pi_k = 0 \to \pi_k = \left(\frac{\lambda}{\mu}\right)^k \pi_0\\
		\vdots \\ \\
	\end{cases} $
	
	Como $ \sum_{i \in S} \pi_i = 1 $, temos que $ \pi_0 \left(1 + \sum_{n=1}^{\infty} \left(\frac{\lambda}{\mu}\right)^n\right) = 1 $. A soma $ \sum_{n=1}^{\infty} \left(\frac{\lambda}{\mu}\right)^n $ só converge quando $ \lambda < \mu $. Quando isso acontece, a soma é finita, e existe a distribuição estacionária.
	
	\section*{Exercício 7}
	\section*{Exercício 8}
	\section*{Exercício 10}
	 
\end{document}