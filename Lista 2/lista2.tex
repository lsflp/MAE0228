\documentclass[12pt,letterpaper]{article}
\usepackage[utf8]{inputenc}
\usepackage{amsmath,amsthm,amsfonts,amssymb,amscd}
\usepackage[table]{xcolor}
\usepackage[margin=2cm]{geometry}
\usepackage{graphicx}
\usepackage{multicol}
\usepackage{mathtools}
\usepackage{palatino}
\usepackage[brazil]{babel}
\newlength{\tabcont}
\setlength{\parindent}{0.0in}
\setlength{\parskip}{0.05in}

\begin{document}
	
	\textbf{Nome}: Luís Felipe de Melo Costa Silva \\
	\textbf{Número USP}: 9297961 
	
	\begin{center}
		\LARGE \bf
		Lista de Exercícios 2 - MAE0228
	\end{center}
	
	\section*{Exercício 2}
	
	\textbf{a)} Sabendo que $E(X) = \int_{-\infty}^{\infty} x \cdot f(x)dx$, pode-se fazer:
	
	\begin{equation*}
		\begin{split}
			E(aX+b) & = \int_{-\infty}^{\infty} (ax+b) \cdot f(x)dx \\
			& = \int_{-\infty}^{\infty} ax \cdot f(x) + b\cdot f(x)dx \\
			& = \int_{-\infty}^{\infty} ax \cdot f(x)dx + \int_{-\infty}^{\infty}b\cdot f(x)dx \\
			& = a \cdot \int_{-\infty}^{\infty} x \cdot f(x)dx + b\cdot \int_{-\infty}^{\infty} f(x)dx \\
			& = a \cdot E(X) + b
		\end{split}
	\end{equation*}
	
	\qed
	
	\textbf{b)} Usando a propriedade $Var(X) = E(X^2) - E^2(X)$ e o resultado do item \textbf{a}:
	
	\begin{equation*}
		\begin{split}
			Var(aX+b) & = E[(aX+b)^2] - E^2(aX+b) \\
			& = \int_{-\infty}^{\infty} (a^2x^2 + 2abx + b^2) \cdot f(x)dx - [\int_{-\infty}^{\infty} (ax+b) \cdot f(x)dx]^2 \\
			& = \int_{-\infty}^{\infty} a^2x^2 \cdot f(x)dx+ \int_{-\infty}^{\infty} 2abx \cdot f(x)dx + \int_{-\infty}^{\infty} b^2 \cdot f(x)dx - [a \cdot E(X) + b]^2 \\
			& = a^2 \cdot \int_{-\infty}^{\infty} x^2 \cdot f(x)dx+ 2ab \cdot \int_{-\infty}^{\infty} x \cdot f(x)dx + b^2 \cdot \int_{-\infty}^{\infty} f(x)dx - [a \cdot E(X) + b]^2 \\
			& = a^2 \cdot E(X^2) + 2ab \cdot E(X) + b^2 - a^2 \cdot E^2(X) - 2ab \cdot E(X) - b^2\\
			& = a^2 \cdot [E(X^2) - E^2(X)] = a^2 \cdot Var(X) 
		\end{split}
	\end{equation*}
	
	\qed
	
	\section*{Exercício 3}
	
	Do enunciado temos que $\lambda = 0.5$ defeito por quilômetro.
	
	\textbf{a)} Para um rolo de $3000$ m ($3$ km), esperamos que existam $1.5$ defeitos. Portanto, $\lambda = 1.5$. A probabilidade de algum defeito é a probabilidade do número de defeitos ser maior do que zero.
	
	\begin{center}
		$P(X>0) = 1 - P(X=0) = 1 - \frac{e^{-1.5}\cdot 1.5^0}{0!} \cong 1 - 0.2231 = 0.7769$
	\end{center} 
	
	\textbf{b)} Analogamente ao item anterior, em um rolo de $6$ km esperamos $3$ defeitos. Portanto, $\lambda = 3$. Queremos a probabilidade de mais de um defeito:
	
	\begin{equation*}
		\begin{split}
			P(X>1) & =   1 - P(X \leq 1) = 1 - P(X=1) - P(X=0) = \\
			& = 1 - \frac{e^{-3}\cdot 3^1}{1!} - \frac{e^{-3}\cdot 3^0}{0!} \cong 1 - 0.1493 - 0.0497 = 0.801\\ 
		\end{split}
	\end{equation*}
	
	\section*{Exercício 6}
	
	Pelo enunciado, é possível inferir que $X$ é uma variável aleatória que segue o modelo de distribuição \textit{Geométrico} (por definição, o caso especial da \textit{Bernoulli} com um sucesso). Portanto, 
	
	\begin{center}
		$P(X=k) = (1-p)^{k-1} \cdot p$
	\end{center}
	
	Vamos calcular $E(X)$. 
	
	\begin{equation*}
		\begin{split}
			E(X) & = \sum_{k=0}^{\infty} k \cdot P(X=k) = \sum_{k=0}^{\infty} P(X>k) = \sum_{k=1}^{\infty} P(X \geq k)\\ 
			& = \sum_{k=1}^{\infty} P(X=k) + \sum_{k=2}^{\infty} P(X=k) + \sum_{k=3}^{\infty} P(X=k) +... \\
			& = \sum_{k=1}^{\infty} (1-p)^{k-1} \cdot p + \sum_{k=2}^{\infty} (1-p)^{k-1} \cdot p + \sum_{k=3}^{\infty} (1-p)^{k-1} \cdot p + ... \\
			& = p \cdot [\sum_{k=1}^{\infty} (1-p)^{k-1} + \sum_{k=2}^{\infty} (1-p)^{k-1} + \sum_{k=3}^{\infty} (1-p)^{k-1} + ...]\\
			& = p \cdot [\frac{1}{1-(1-p)} + \frac{1-p}{1-(1-p)} + \frac{(1-p)^2}{1-(1-p)} + ...]\\
			& = p \cdot [\frac{1}{p} + \frac{1-p}{p} + \frac{(1-p)^2}{p} + ...]\\
			& = 1 + (1-p) + (1-p)^2 + ... = \sum_{i = 0}^{\infty} (1-p)^i = \frac{1}{1-(1-p)} = \frac{1}{p}\\
		\end{split}
	\end{equation*}
	
	\section*{Exercício 8}
	
	$E(\frac{1}{X+1})$ será calculada usando a definição de esperança para variáveis aleatórias discretas, como é no caso da Poisson. Logo:
	
	\begin{equation*}
		\begin{split}
			E(\frac{1}{X+1}) & = \sum_{x=0}^{\infty} \frac{1}{x+1} \cdot P(X=x)\\
			& = \sum_{x=0}^{\infty} \frac{1}{x+1} \cdot \frac{e^{-\lambda}\cdot \lambda^x}{x!} = \sum_{x=0}^{\infty} \frac{e^{-\lambda}\cdot \lambda^x}{(x+1)!} \\
			& = \sum_{x=0}^{\infty}\frac{1}{\lambda} \cdot \frac{e^{-\lambda}\cdot \lambda^{x+1}}{(x+1)!} = \frac{1}{\lambda} \cdot \sum_{x=0}^{\infty} \frac{e^{-\lambda}\cdot \lambda^{x+1}}{(x+1)!} \\
			& = \frac{1}{\lambda} \cdot \sum_{y=1}^{\infty} \frac{e^{-\lambda}\cdot \lambda^y}{y!} \text{ , onde $y=x+1$}\\
			& = \frac{1}{\lambda} \cdot [1-P(Y=0)] \text{ , onde Y$\sim$Poisson($\lambda$)} \\ 
			& = \frac{1}{\lambda} \cdot (1 - e^{-\lambda}) = \frac{1 - e^{-\lambda}}{\lambda} 
	\end{split}
	\end{equation*}
	
	
\end{document}