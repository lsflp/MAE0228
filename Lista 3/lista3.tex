\documentclass[12pt,letterpaper]{article}
\usepackage[utf8]{inputenc}
\usepackage{amsmath,amsthm,amsfonts,amssymb,amscd}
\usepackage[table]{xcolor}
\usepackage[margin=2cm]{geometry}
\usepackage{graphicx}
\usepackage{multicol}
\usepackage{mathtools}
\usepackage{palatino}
\usepackage[brazil]{babel}
\newlength{\tabcont}
\setlength{\parindent}{0.0in}
\setlength{\parskip}{0.05in}

\begin{document}
	
	\textbf{Nome}: Luís Felipe de Melo Costa Silva \\
	\textbf{Número USP}: 9297961 
	
	\begin{center}
		\LARGE \bf
		Lista de Exercícios 3 - MAE0228
	\end{center}
	
	\section*{Exercício 2}
	
	Para esse exercício, vamos considerar:
	
	\begin{itemize}
		\item $T_A$ como o tempo de atendimento de $A$.
		\item $T_B$ como o tempo de atendimento de $B$.
		\item $T_C$ como o tempo de atendimento de $C$.
	\end{itemize}
	
	\textbf{a)} Queremos calcular $P(T_A > T_B + T_C)$. Aqui, os tempos são determinísticos, então, $T_A = T_B = T_C = 5$. Logo, $P(T_A > T_B + T_C) = P(5 > 10) = 0$. \\
	
	\textbf{b)} Vamos calcular a distribuição de $T_B+T_C = T_{B+C}$. O valor de $T_B$ está nas linhas, e o de $T_C$, nas colunas. Os valores do interior são os possíveis para $T_{B+C}$.
	
	\begin{center}
		\begin{tabular}{|l|l|l|l|l}
			\cline{1-4}
			$T_B+T_C$ & 1 & 2 & 3 &  \\ \cline{1-4}
			1         & 2 & 3 & 4 &  \\ \cline{1-4}
			2         & 3 & 4 & 5 &  \\ \cline{1-4}
			3         & 4 & 5 & 6 &  \\ \cline{1-4}
		\end{tabular}
	\end{center}
	
	Logo, assumindo independência entre $T_A$, $T_B$ e $T_C$ (já que o tempo de cada caixa é regido por uma variável aleatória, e portanto, o tempo de atendimento de cada cliente será um valor):
	
	\begin{equation*}
		\begin{split}
			P(T_A > T_{B+C}) & = \sum_{T_A = 1}^{3} \sum_{T_{B+C} = 1}^{T_A-1} P(T_A) \cdot P(T_{B+C}) = \\
			& = P(T_A = 2) \cdot P(T_{B+C} = 1) + P(T_A = 3) \cdot P(T_{B+C} = 1) + P(T_A = 3) \cdot P(T_{B+C} = 2) = \\
			& = P(T_A = 3) \cdot P(T_{B+C} = 2) = \frac{1}{3} \cdot \frac{1}{9} = \frac{1}{27}
		\end{split}
	\end{equation*}
	
	\textbf{c)} Fazendo $T_A = W$, $T_B = X$, $T_C = Y$, $T_B+T_C=T_{B+C} = Z$, vamos achar a distribuição de Z, sabendo que $f_x(x) = \mu \cdot e^{-\mu \cdot x}$ e que $f_y(y) = \mu \cdot e^{-\mu \cdot y}$.
	
	\begin{equation*}
		\begin{split}
			F_z(z) & = \int_{0}^{z} \int_{0}^{z-y} \mu \cdot e^{-\mu \cdot x} \cdot \mu \cdot e^{-\mu \cdot y} \text{d}x \text{d}y = \int_{0}^{z}  \mu \cdot e^{-\mu \cdot y} \cdot \left[-e^{-\mu \cdot x}\right]_{0}^{z-y} \text{d}y = \\
			& = \int_{0}^{z}  \mu \cdot e^{-\mu \cdot y} \cdot \left[1 - e^{-\mu(z-y)}\right] \text{d}y = \int_{0}^{z}  \mu \cdot e^{-\mu \cdot y} - \mu \cdot e^{-\mu \cdot z} \text{d}y = \\
			& = \left[-e^{-\mu \cdot y} - y \cdot \mu \cdot e^{-\mu \cdot z}  \right]_{0}^{z} = -e^{-\mu \cdot z} - z \cdot \mu \cdot e^{-\mu \cdot z} - (-1 - 0 \cdot \mu \cdot e^{-\mu \cdot z}) = \\
			& = 1 -e^{-\mu \cdot z} - z \cdot \mu \cdot e^{-\mu \cdot z} 
		\end{split}
	\end{equation*}
	
	Portanto, 
	
	\begin{equation*}
		\begin{split}
			f_z(z) = F_z'(z) = \mu \cdot e^{-\mu \cdot z} - \mu \cdot e^{-\mu \cdot z} + z \cdot \mu^2 \cdot e^{-\mu \cdot z} = z \cdot \mu^2 \cdot e^{-\mu \cdot z} = \frac{z^{2-1} \cdot \mu^2 \cdot e^{-\mu \cdot z}}{\Gamma(2)}
		\end{split}
	\end{equation*}
	
	Logo, $Z\sim$ Gama$(2,\mu)$. Agora, vamos calcular $P(W>Z)$, assumindo independência, assim como no exercício anterior, entre $W$ e $Z$: 
	
	\begin{equation*}
		\begin{split}
			P(W>Z) & = \int_{0}^{\infty} \int_{0}^{w} f_w(w) \cdot f_z(z) \text{d}z \text{d}w = \int_{0}^{\infty} \int_{0}^{w} \mu \cdot e^{-\mu \cdot w} \cdot z \cdot \mu^2 \cdot e^{-\mu \cdot z} \text{d}z \text{d}w = \\
			& = \int_{0}^{\infty} \int_{0}^{w} \mu \cdot e^{-\mu \cdot w} \cdot f_z(z) \text{d}z \text{d}w = \int_{0}^{\infty} \mu \cdot e^{-\mu \cdot w} \cdot \int_{0}^{w} f_z(z) \text{d}z \text{d}w = \\
			& = \int_{0}^{\infty} \mu \cdot e^{-\mu \cdot w} \cdot F_z(w) \text{d}w = \int_{0}^{\infty} \mu \cdot e^{-\mu \cdot w} \cdot [1 -e^{-\mu \cdot w} - w \cdot \mu \cdot e^{-\mu \cdot w} ] \text{d}w = \\
			& = \int_{0}^{\infty} \mu \cdot e^{-\mu \cdot w} \text{d}w - \int_{0}^{\infty} \mu \cdot e^{-2 \cdot\mu \cdot w} \text{d}w - \int_{0}^{\infty} w \cdot \mu^2 \cdot e^{-2 \cdot \mu \cdot w} \text{d}w  = \\
			& = \int_{0}^{\infty} \mu \cdot e^{-\mu \cdot w} \text{d}w - \frac{1}{2} \int_{0}^{\infty} 2\cdot \mu \cdot e^{-2 \cdot\mu \cdot w} \text{d}w - \frac{1}{4}\int_{0}^{\infty} \frac{w \cdot 4 \cdot \mu^2 \cdot e^{-2 \cdot \mu \cdot w}}{\Gamma(2)} \text{d}w = \\
			& = 1 \cdot 1 - \frac{1}{2} \cdot 1 - \frac{1}{4} \cdot 1 = \frac{1}{4}
		\end{split}
	\end{equation*}
	
	\section*{Exercício 5}
	
	\textbf{a)} Na $i$-ésima iteração, a probabilidade de o número analisado ser o maior do conjunto é de $\frac{1}{i}$, pois, com $i$ números, apenas 1 deles é o maior. \\
	
	\textbf{b)} A linha $5$ só pode ser executada $0$ ou $1$ vez. Ela será executada quando o número analisado for o maior do conjunto, ou seja, $P(X_i = 1) = \frac{1}{i}$, e, portanto, $P(X_i = 0) = \frac{i-1}{i}$. Logo, $X_i\sim Bernoulli(\frac{1}{i})$ e:
	
	\begin{center}
		$E(X_i) = \sum_{k=0}^{1} k \cdot P(X=k) = 0 \cdot P(X=0) + 1 \cdot P(X=1) = P(X=1) = \frac{1}{i}$
	\end{center}
	
	\textbf{c)} Para que $X_i$ seja independente de $X_j$, $P(X_j | X_i) = P(X_j)$.
	
	Vamos analisar $P(X_j | X_i)$. Seja $i < j$. $P(X_j | X_i)$ é a probabilidade de o $j$-ésimo número ser o maior no conjunto de $j$ números dado que o $i$-ésimo número era o maior do conjunto de $i$ números. Portanto, isso é a probabilidade de o $j$-ésimo ser o maior no conjunto de números entre as posições $i$ e $j$, ($i$ e $j$ incluídos). Como existem $j-i+1$ números nesse conjunto, então, $P(X_j | X_i) = \frac{1}{j-i+1} \neq P(X_j) = \frac{1}{j}$. 
	
	Como pode-se ver, os $X_i$ não são independentes. A distribuição deles é a mesma (Bernoulli). No entanto, o parâmetro é diferente para cada variável, o que mostra que elas não são identicamente distribuídas. \\
	
	\textbf{d)} Calculando a $E(T)$:
	
	\begin{equation*}
		\begin{split}
		E(T) & = E\left(\sum_{i=0}^{n} X_i\right) = \sum_{i=0}^{n} E(X_i) = \sum_{i=0}^{n} \frac{1}{i} = \\
		& = 1 + \frac{1}{2} + \frac{1}{3} + ... + \frac{1}{n} \leq \int_{1}^{n} \frac{1}{x} \text{ d}x = \text{ln } n
		\end{split}
	\end{equation*}
	
	Portanto, $E(T) = \mathcal{O}(\text{log } n)$.
	
	\section*{Exercício 6}
	
	\textbf{b)} Sabendo, do item anterior, que $X+Y \sim Poisson(\lambda + \mu)$ e que $X$ e $Y$ são independentes, teremos: 
	
	\begin{equation*}
		\begin{split}
			P(X|X+Y=n) & = \frac{P(X=x, X+Y=n)}{P(X+Y=n)} = \frac{P(X=x, Y=n-x)}{P(X+Y=n)} = \frac{P(X=x) \cdot P(Y=n-x)}{P(X+Y=n)} =\\
			& = \left[\frac{e^{-\lambda} \cdot \lambda^x}{x!} \cdot \frac{e^{-\mu} \cdot \mu^{n-x}}{(n-x)!}\right] \cdot \frac{n!}{e^{-(\lambda+\mu)}\cdot(\lambda+\mu)^n} = \\
			& = \frac{n!}{x!\cdot(n-x)!} \cdot \frac{e^{-(\lambda+\mu)}}{e^{-(\lambda+\mu)}} \cdot \frac{\lambda^x \cdot \mu^{n-x}}{(\lambda+\mu)^n} = \binom{n}{x} \frac{\lambda^x}{(\lambda+\mu)^x} \cdot \frac{\mu^{n-x}}{(\lambda+\mu)^{n-x}} = \\
			& = \binom{n}{x} \left(\frac{\lambda}{\lambda+\mu}\right)^x \cdot \left(\frac{\mu}{\lambda+\mu}\right)^{n-x} \sim Binomial\left(n, \frac{\lambda}{\lambda+\mu}\right)\\
		\end{split}
	\end{equation*}
	
	\qed
	
	\section*{Exercício 8}
	
	Acatando a sugestão do enunciado, vamos particionar $X$ como $X = A+B+C$, onde:
	
	\begin{itemize}
		\item $A$ é o número de retiradas até o primeiro cupom do primeiro tipo.
		\item $B$ é o número de retiradas a partir do cupom de primeiro tipo até o primeiro cupom de um tipo diferente.
		\item $C$ é o número de retiradas a partir do momento em que temos dois cupons de tipos diferentes até o primeiro cupom do terceiro tipo.
	\end{itemize}
	
	As retiradas são feitas com reposição. Então, teremos as seguintes definições para cada uma das variáveis aleatórias acima:
	
	\begin{itemize}
		\item $A \sim$ Geométrica($1$). $E(A) = \frac{1}{1} = 1$ e $Var(A) = \frac{1-1}{1^2} = 0$
		\item $B \sim$ Geométrica$\left(\frac{9}{10}\right)$. $E(B) = \frac{1}{\frac{9}{10}} = \frac{10}{9} =1.\overline{1}$ e $Var(B) = \frac{1-\frac{9}{10}}{(\frac{9}{10})^2} \cong 0.12346$
		\item $C \sim$ Geométrica$\left(\frac{8}{10}\right)$. $E(C) = \frac{1}{\frac{8}{10}}= \frac{10}{8} = 1.25$ e $Var(C) = \frac{1-\frac{8}{10}}{(\frac{8}{10})^2} = 0.3125$
	\end{itemize}
	
	É importante notar que $A$, $B$ e $C$ são \textbf{independentes} umas das outras. Como as retiradas são feitas \textbf{com reposição}, um evento não interfere no seu sucessor. Quando fazemos as $A$ retiradas para o cupom do primeiro tipo, começaremos as $B$ retiradas para o cupom do segundo tipo, que será um novo experimento independente do anterior, com probabilidade diferente. Para o terceiro tipo de cupom, vale o mesmo raciocínio. Logo:
	
	\begin{itemize}
		\item $E(X) = E(A+B+C) = E(A) + E(B) + E(C) = 3.36\overline{1}$
		\item $Var(X) = Var(A+B+C) = Var(A) + Var(B) + Var(C) \cong 0.43596$
	\end{itemize}
			
	 
\end{document}